PHP is one of the most popular languages for server-side web-development. It is powering over 80\% of the web\citeB{w3techs}, including major websites, such as Wikipedia and Facebook, and the most used content management systems: Wordpress, Joomla, Drupal, and Magento. It requires no compilation and is dynamically typed, which makes development and deployment easy. 

As with other dynamically typed languages, static type reasoning is non-trivial. This task is only worsened by allowing variable-variables, variable-functions and the all-purpose array datastructure. PHP supports associative arrays with integer and/or string-typed indices (referred to as keys) mapping to values of arbitrary type, including arrays. Combined with the dynamic type system, extensive coercion, and optional error-reporting, debugging becomes difficult and time consuming. Furthermore, no official language specification exists and the language is thus defined by the reference Zend Engine interpreter.


This thesis will focus on a static type analysis of arrays. Reasoning about the structure of an array with a decent level of precision seems like an impossible task, since practically no structure is imposed on arrays. But is the imagination of the average PHP developer in practice limited? Are arrays used as other data-structures, such as maps and lists, and can these structures be identified statically? If this is the case, then these structures might be the key to an abstraction yielding a fair compromise between speed and precision for a static type analysis. 

By analysing a corpus of existing frameworks dynamically, this thesis aims to identify use-patterns of arrays as other, more restrictive, data-structures. The results of the dynamic analysis are going to motivate the data-abstraction in a \emph{static interprocedural data-flow type analysis} on a subset of the PHP language. The static analysis should serve as a tool in a development environment, facilitating error detection by identifying suspicious code. 

%As mentioned before, error-reporting is optional, and throwing a warning or a notice might not stop the program from 

The following example, found in the Part framework (a CMS written by one of the authors), illustrates suspicious usage of arrays. Here the \texttt{\$keyArray} and \texttt{\$valueArray} arrays are first used as maps in lines \ref{line:part-keyar1} and \ref{line:part-valar1}, while later, at line \ref{line:part-keyar2} and \ref{line:part-valar2} being used as lists, with the \emph{array append} operation. The intention of this kind of usage is unclear and not very maintainable, but since it ultimately results in the correct behavior, and yields no errors, it is not easily discovered. The analysis should facilitate discovery of such suspicious cases by providing feedback to the user in the development environment. By bringing the user's attention to suspicious cases while developing it is possible to further the clear intention of the code thereby creating more maintainable code.

\begin{program}
\centering 
\begin{lstlisting}
private function createInstance($string, $instance, callable $callback)
{
    if (!isset($this->keyArray[$string])) {
        $this->keyArray[$string] = []; %*\label{line:part-keyar1}*)
        $this->valueArray[$string] = []; %*\label{line:part-valar1}*)
    } else if(($k = array_search($instance, $this->keyArray, true)) !== false){
        return $this->valueArray[$k];
    }
    $this->keyArray[] = $instance; %*\label{line:part-keyar2}*)
    return $this->valueArray[] = $callback(); %*\label{line:part-valar2}*)
}
\end{lstlisting}
\caption{Mixing array types}
\end{program}

\section{Problem statement}
The widely used array data structure in PHP has very few restrictions, making it difficult to reason about with program analysis. This thesis will identify possible use-patterns for arrays in PHP and how to detect them using a static analysis. An \emph{interprocedural data-flow type-analysis} is proposed to detect suspicious cross-use of the identified patterns, and consequently evaluated on live code.

\section{Motivation}
Creating code that seemingly works is one thing, and for languages like PHP that can be done in many different ways. However creating code that conveys a clear purpose and is easily maintainable later on is a whole other world. The latter is preferable most of the time. Due to the range of possible uses of PHP arrays being very large, they are not conveying a clear purpose in and of themselves. Creators of PHP programs thus have to ensure the clarity of their use of arrays themselves. 

We hypothesize that arrays are subject to a specific use-pattern during their lifetime. If the hypothesis holds, a statical analysis can be employed to detect and thereby prevent cross-use of the patterns, which in turn will lead to a more clear intention of the code and in the end higher maintainability of the program code.

Introducing feedback from the analysis directly in the integrated development environment used in daily work ensures a small overhead for the benefits gained by using feedback to develop better programs.

\section{Structure of this thesis}
Chapter \ref{ch:background} provides the necessary background knowledge of the PHP language and modern PHP program structure to understand, why arrays are difficult to apply existing methods to. In chapter \ref{ch:study}, a dynamic analysis is conducted on a corpus of real PHP applications, to identify use-patterns and test the hypothesis about use-patterns of arrays. In chapter \ref{ch:analysis}, use-patterns and knowledge gained in the previous chapter is utilized to define and implement a static analysis of a subset of PHP, focusing on detecting suspicious use of arrays. The static analysis implementation is evaluated in chapter \ref{ch:evaluation} by studying interesting cases found in or inspired by the corpus used in the dynamic analysis. Related work is discussed in chapter \ref{ch:relatedWork}. The last two chapters, \ref{ch:conclusion} and \ref{ch:futureWork}, conclude our thesis and describe possible future work.
