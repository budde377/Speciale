
% second attempt
The PHP array data-structure has very few limits. Due to the many possibilities, array usage can obscure the intention of code, thereby making it difficult to maintain. Keeping precision about data in arrays is difficult and resource consuming. By utilizing usage patterns identified in a dynamic analysis, this thesis develops a light-weight array data abstraction for use in a classic static interprocedural data-flow analysis for a defined subset of the PHP language. The analysis is evaluated with case studies of small programs inspired by flaws found in the corpus applications. We expect the feedback provided by the analysis to enable the user to write code with a clearer intention and thus a higher maintainability.

% first attempt
\begin{comment}
Writing code with a clear intention and high maintainability is important for any development project. For the PHP language the many possibilities of the array data-structure can easily hinder those important properties. Even if PHP arrays have endless possible use cases, does the practical uses not exhibit detectable patterns? This thesis aims to answer that question with a dynamical analysis of a corpus of popular PHP applications. Using the patterns identified in the dynamical analysis a data abstraction is developed for use in a classical static interprocedural data-flow analysis for a defined subset of the PHP language. The analysis is evaluated with case studies of small programs inspired by flaws found in the corpus applications. The case studies show how the analysis can provide feedback for the user when code is produced that might hinder the clarity of intention or maintainability.
\end{comment}