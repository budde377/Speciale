\section{The Monotone Framework}
\label{sec:framework}
In order to perform a data flow analysis on a P0 program, the Embellished Monotone Framework, as introduced by Nielson, Nielson and Hanikin in \citeA{popa}, is used. This section describes how the concepts of the previous sections fit into the framework. A \emph{monotone data-flow analysis} is a tuple $(L, \mathcal{F})$ where $L$ is a lattice and $\mathcal{F}$ is a monotone function space on $L$.

Given a program, a corresponding instance of the data-flow analysis can be expressed as a six-tuple $(L, \mathcal{F}, F, E', \iota, f)$. Here $L$ and $\mathcal{F}$ are the lattice and function space of the analysis, $F$ is a set of tuples expressing the flow between labels, $E'$ is the set of external labels, $\iota$ is the initial lattice element of the external labels, and $f$ is a mapping from labels to functions in $\mathcal{F}$. Labels are in this analysis defined as the product of nodes and context, $\mathcal{L} = \mathcal{N}\times \Delta$. From an instance, the data-flow equations $A = \mathcal{L} \rightarrow L$ are defined as
\begin{align}
	A_\bullet ((n, \delta)) = \begin{cases} 
		f_{(n, \delta, (\delta, c))}(A_\circ(( c, \delta)), A_\circ((n, \delta))) &\text{if $n=\mathit{return}_c(\_)$}\\
		f_{(n, \delta)}(A_\circ((n, \delta)))& \text{else}
		\end{cases}
\end{align}
where 
\begin{align}
	A_\circ (l) = \bigsqcup \{A_\bullet(l') | (l', l) \in F\} \sqcup \iota_{E'}^l
\end{align}
and
\begin{align}
\iota_{E'}^l = \begin{cases} \iota & \text{if }l\in E' \\  \bot &\text{else} \end{cases}
\end{align}
Intuitively, $A_\bullet$ expresses the abstract state immediately after performing the given label and $A_\circ$ the state just before. Solving these equations performs the data-flow analysis. 

Given a control-flow graph $G = (V,E,s,t)$, an instance of the analysis can be derived, where $L = \text{AnalysisLattice}$ and $\mathcal{F}$ is the function space of the transfer functions. The mapping $f$ is defined throughout section \ref{sec:transferf} mapping nodes to transfer functions. $F : \mathcal{L} \times \mathcal{L}$ is defined as 
\begin{align}
F = &\{((n, \delta), (n', \delta')) | (n,n') \in E \wedge\delta \in \Delta \notag\\
&\quad \wedge \mathit{validSuccessor}(n, n')\notag\\
&\quad \wedge  \delta' = \mathit{nextC}( n, \delta)\}
\end{align} 
where 
\begin{align}
\mathit{nextC}(n,\delta) = 
	\begin{cases}
		\delta' & \text{if $n = \mathit{exit(\_)}$ and $\delta = (\delta', c)$}\\
		(\delta, n) &\text{if }n=\mathit{call}(\_)\\
		\delta &\text{else} 
	\end{cases}
\end{align} The predicate $\mathit{validSuccessor}$ is defined by definition \ref{def:validsuc}, and it is necessary because of the ambiguity associated with potentially multiple outgoing edges of the $\mathit{exit}$ node, which indicates that the flow may go from an exit node to an arbitrary result node. This is naturally not the case. A successor $w$ to a node $v$ is valid if and only if $v$ is an exit node and $w$ is the return node corresponding to the call node of the current function-call, or if $v$ is not an exit node.

\begin{definition}
\label{def:validsuc}
The predicate $\mathit{validSuccessor} : \mathcal{L} \times \mathcal{L}$, is defined as
\begin{align}
\mathit{validSuccessor}(n, \delta, n', \delta') \Leftrightarrow &(n = \mathit{exit}(\_) \wedge n' = \mathit{result}_c(\_) \wedge \delta  = (\delta' c)) \notag\\
&\quad\vee n \neq \mathit{exit}(\_)
\end{align}
\end{definition}


The set of external nodes can be defined as $E' = \{s\}$. The initial lattice-element, $\iota$, is the element where the empty context, $\Lambda$, maps to the state, $s_\iota = (\bot, g_\iota, h_\iota, \bot, \bot)$. Here the global scope $g_\iota$ models the superglobals as
\begin{align*}
g_\iota = 	[&\texttt{\$GLOBALS}\mapsto \{\text{HLoc}(\Lambda, s, 0)\},\notag\\
			 &\texttt{\$\_SERVER}\mapsto \{\text{HLoc}(\Lambda, s, 1)\},\notag\\
			 &\texttt{\$\_SESSION}\mapsto \{\text{HLoc}(\Lambda, s, 2)\},\notag\\
			 &\texttt{\$\_ENV}\mapsto \{\text{HLoc}(\Lambda, s, 3)\},\notag\\
			 &\texttt{\$\_COOKIE}\mapsto \{\text{HLoc}(\Lambda, s, 4)\},\notag\\
			 &\texttt{\$\_POST}\mapsto \{\text{HLoc}(\Lambda, s, 5)\},\notag\\
			 &\texttt{\$\_GET}\mapsto \{\text{HLoc}(\Lambda, s, 6)\},\notag\\
			 &\texttt{\$\_REQUEST}\mapsto \{\text{HLoc}(\Lambda, s, 7)\}, \notag\\
			 &\texttt{\$\_FILES}\mapsto \{\text{HLoc}(\Lambda, s, 8)\}]
\end{align*} 
and the heap, $h_\iota$ is
\begin{align*}
h_\iota = [	&\text{HLoc}(\Lambda, s, i\in[0,2])\mapsto \text{Value}(\text{Array}(\top)),\notag\\
			&\text{HLoc}(\Lambda, s, 3)\mapsto \text{Value}(\text{ArrayMap}([\text{Index}(\top)\mapsto\text{HLoc}(\Lambda, s, 9)])),\notag\\
			&\text{HLoc}(\Lambda, s, 4)\mapsto \text{Value}(\text{ArrayMap}([\text{Index}(\top)\mapsto\text{HLoc}(\Lambda, s, 10)])),\notag\\
			&\text{HLoc}(\Lambda, s, 5)\mapsto \text{Value}(\text{ArrayMap}([\text{Index}(\top)\mapsto\text{HLoc}(\Lambda, s, 11)])),\notag\\			
			&\text{HLoc}(\Lambda, s, 6)\mapsto \text{Value}(\text{ArrayMap}([\text{Index}(\top)\mapsto\text{HLoc}(\Lambda, s, 12)])),\notag\\			
			&\text{HLoc}(\Lambda, s, 7)\mapsto \text{Value}(\text{ArrayMap}([\text{Index}(\top)\mapsto\text{HLoc}(\Lambda, s, 13)])),\notag\\			
			&\text{HLoc}(\Lambda, s, 8)\mapsto \text{Value}(\text{ArrayMap}([\text{Index}(\top)\mapsto\text{HLoc}(\Lambda, s, 14)])),\notag\\			
			&\text{HLoc}(\Lambda, s, i \in [9, 10])\mapsto \text{Value}(\text{String}(\top)),\notag\\
			&\text{HLoc}(\Lambda, s, i\in[11,13])\mapsto \text{Value}(\text{Array}(\top),\text{String}(\top)),\notag\\
			&\text{HLoc}(\Lambda, s, 14)\mapsto \text{Value}(\text{ArrayMap}([\text{Index}(\top)\mapsto\text{HLoc}(\Lambda, s, 15)])),\notag\\			
			&\text{HLoc}(\Lambda, s, 15)\mapsto \text{Value}(\text{Number}(\top),\text{String}(\top)),\notag\\
			&\text{HLoc}(\Lambda, s, 16)\mapsto \text{Value}(\top)\notag\\
			&\_ \mapsto \text{Value}(\text{Null}(\top))]
\end{align*}
The last entry maps all other locations to the null value.

The initial values of the superglobals are based on inspections of the structure of each individual variable. Take the \texttt{\$\_FILES} array as an example. The structure is a map from form-element names to maps of information about the file such as name, size, and type. In the initial lattice-element this structure is expressed by a map of maps containing strings and numbers.  Location $16$ is not used directly in the global scope. The location is created to ensure any initial values of the superglobals are covered when reading from a $\top$-element array. For example reading from the \texttt{\$\_SERVER} superglobal may result in any value defined by the server, which is expressed by the $\top$-element value in the heap. 

By modifying the initial lattice-element it is possible to introduce assumptions about the superglobals into the analysis. By introducing such assumptions it is possible to analyse how different user input and environment will affect a program. The element chosen here is chosen to be sound for all possible values of the superglobals.

