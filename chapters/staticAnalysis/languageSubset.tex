\section{PHP Language subset}
\newcommand{\syn}[1]{\langle#1\rangle}
\label{sec:langsubset}
To simplify the static analysis a subset of the PHP language, P0, is used. The PHP language has no formal definition and thus the language used here can not be formally shown to be a subset of the complete PHP language. The full PHP language is defined by the Zend Interpreter. A formal definition of the language used in this thesis can be found below.

To simplify the analysis and keep the focus on arrays; resource handles and objects have been completely removed from the language. Dynamic dispatch (variable function names), variable variables and dynamic loading of code (\texttt{require} and \texttt{include}) have also been removed. There are no anonymous functions and a limited number of statements. The language does however support all reference features, i.e. reference assigning entries in an array or variables.

The syntax can be expressed with grammar \ref{gramm:p0}. Here $e : \syn{expr}$ denotes an expression, $e : \syn{rexpr}$ a reference expression, and $e : \syn{vexpr}$ a variable expression.   Furthermore it being a subset of PHP, a program is only valid in P0  if it is also a valid PHP program. E.g. while the syntax allows negation of arrays, this action yields a fatal error in PHP, i.e. is an invalid program, hence is also an invalid P0 program. 

Notice that returning an $\syn{rexpr}$ with a non-reference function might result in a fatal error, e.g. when returning the result of an array-append operation, but the same operation is valid in a reference-function.  

Return statements are only allowed in a function body.

Every function ends with a return statement.

\todo{elaborate}


\todo{Write about support of alias}

\begin{grammarf}
\centering
\begin{grammar}
<program> ::= (<function-definition> | <statement>)*
\end{grammar}
\begin{grammar}
<function-definition> ::= `function' `&'?<function-name> `(' (`&'? <var> ( `,' `&'? <var>)* ) | $\epsilon$`)' <block>
\end{grammar}
\begin{grammar}
<statement> ::= `while (' <expr> `)' <statement>
%\alt `foreach (' <expr> `as' <opt-arrow-var> `)' <statement>
\alt `for (' <expr>?`;'<expr>?`;'<expr>? `)' <statement>
\alt `if('<expr>`)' <statement>
\alt `if('<expr>`)' <statement> `else' <statement>
\alt `;'
\alt <expr>`;'
\alt `global' <var> (`,' <var>)*`;'
\alt `return' (<expr>|<rexpr>)?`;'
\alt <block>
\end{grammar}
\begin{grammar}
<expr> ::= <expr> $\oplus$ <expr>
\alt $\circ$ <expr>
\alt `(' <expr> `)'
\alt <vexpr>`+ +'
\alt <vexpr>`- -'
\alt `+ +'<vexpr>
\alt `- -'<vexpr>
\alt <var>
\alt <expr> `['<expr> `]'
\alt <function-reference>
\alt <const>
\alt <assignment>
\alt `[' ($\epsilon$ | <array-init-entry> (`,' <array-init-entry>)*) `]'
\alt `array(' ($\epsilon$ | <array-init-entry> (`,' <array-init-entry>)*) `)'
\end{grammar}
\begin{grammar}
<function-reference> ::= <function-name>`(' (<function-arg> (`,' <function-arg>)* | $\epsilon$ ) `)'
\end{grammar}
\begin{grammar}
<function-arg> ::= <expr> 
\alt <rexpr> 
\end{grammar}
\begin{grammar}
<rexpr> ::= <var> 
\alt <function-reference>
\alt <rexpr>`[]'
\alt <rexpr>`[' <expr> `]'
\end{grammar}
\begin{grammar}
<vexpr> ::= <var> 
\alt <rexpr>`[]'
\alt <rexpr>`[' <expr> `]'
\end{grammar}
\begin{grammar}
<array-init-entry> ::= <expr> `=>' <expr> 
\alt <expr>
\end{grammar}
\begin{grammar}
<assignment> ::= <rexpr>`[]'  `=' <expr>`;'
\alt <rexpr>`[' <expr> `]'  `=' <expr>`;'
\alt <var> `=' <expr>
\alt <rexpr>`[]'  `=' `&' <rexpr>`;'
\alt <rexpr>`[' <expr> `]'  `=' ` &' <rexpr>`;'
\alt <var> `=' `&' <rexpr>
\end{grammar}

\begin{grammar}
<block> ::= `{' <statement>* `}'
\end{grammar}
\caption{P0 syntax }
\label{gramm:p0}
\end{grammarf}