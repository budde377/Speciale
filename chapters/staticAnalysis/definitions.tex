
\newcommand{\pvar}[1]{#1}
\newcommand{\px}{x}
\newcommand{\pint}{\texttt{int}}
\newcommand{\pfloat}{\texttt{float}}
\newcommand{\pscalar}{\texttt{scalar}}
\newcommand{\pbool}{\texttt{boolean}}
\newcommand{\pstring}{\texttt{string}}
\newcommand{\parray}{\texttt{array}}
\newcommand{\pcallable}{\texttt{callable}}
\newcommand{\pamap}[1][\tau]{\texttt{aMap<$#1$>}}
\newcommand{\paobj}{\texttt{aObj}}
\newcommand{\palst}[1][\tau]{\texttt{aList<$#1$>}}
\newcommand{\paslst}[1][\tau]{\texttt{aSparseList<$#1$>}}
\newcommand{\presource}{\texttt{resource}}
\newcommand{\pobject}{\texttt{object}}
\newcommand{\pnull}{\texttt{null}}

\section{Types}
\todo{Assume law of excluded middle?}
\begin{definition}
Let $\mathcal{C}$ denote the set of all classes and interfaces. 
\end{definition}
\todo{example}
\begin{definition}
The set of all possible types as \pint, \pfloat, \pbool, \pstring, \pscalar, \parray, \presource, \pcallable, \pobject{} and all $c \in \mathcal{C}$.
\end{definition}


Given a variable \pvar{x} we denote the type, $\tau$ as $\pvar{x} : \tau $. A given variable can have multiple types E.g. since scalars are numbers, booleans, strings and null, any variable that is an integer is also a scalar. This can be expressed with the following rules 

\begin{definition} 
Subtyping
\begin{align}
\pint &<:  \pscalar \\
\pfloat &<: \pscalar \\
\pbool &<:  \pscalar \\
\pstring &<: \pscalar
\end{align}
The \pobject{} type can be viewed as an equivalent of the \texttt{Java.lang.Object} class, which gives 
\begin{align}
\tau <: \pobject
\end{align}
where $\tau \in \mathcal{C}$
\end{definition}

In order to reason about arrays as objects, lists and maps, four more types are introduced : \pamap, \palst{}, \paslst{} and \paobj. Here $\tau$ is a type. 

\begin{definition} The \emph{isList},  \emph{isSparseList}, \emph{isMap} and \emph{isObject} relations, where $\tau$ is some type and \px{} : \parray, are defined as

\begin{align}
    isList[\tau](\px) &\Leftrightarrow listkeys(\px) \wedge vht[\tau](\px)\\
    isSparseList[\tau](\px) &\Leftrightarrow  \forall (k,\_) \in \px.\;k : \pint \wedge isMap[\tau](\px)\\
    isMap[\tau](\px) &\Leftrightarrow \neg listkeys(\px) \wedge vht[\tau](\px)\\
    isObject(\px) &\Leftrightarrow \forall \tau. \neg vht[\tau](\px)
\end{align}
where 
\begin{align}
listKeys(\px) &\Leftrightarrow \forall i \in [0, count(\px)).\;(i, \_) \in x\\
vht[\tau](\px) & \Leftrightarrow \forall (\_, v_1),(\_, v_2) \in \px.\; v_1 : \tau \wedge v_2 : \tau
\end{align}
\emph{count} returns the size of the array.
\end{definition}

\begin{definition}
Let \px{} be some variable of type \parray{}
\begin{align}
\exists \tau. isMap[\tau](\px) &\Leftrightarrow \px : \pamap \label{eq:defMapMap}\\
\exists \tau. isList[\tau](\px) &\Leftrightarrow \px : \palst\label{eq:defLstLst} \\
\exists \tau. isSparseList[\tau](\px) &\Leftrightarrow \px : \paslst\label{eq:defLstLst} \\
isObject(\px) &\Leftrightarrow \px : \paobj \label{eq:defObjObj}
\end{align}

Let $\tau$ be some type then it holds that
\begin{align}
    \pamap & <: \parray\\
    \palst & <: \parray\\
    \paobj & <: \parray\\
    \paslst & <: \pamap
\end{align}
\begin{comment}
\begin{prooftree}
\AxiomC{$S <: T$} \AxiomC{$T <: S$}
\BinaryInfC{$\palst[S] <: \palst[T]$}
\end{prooftree}
\begin{prooftree}
\AxiomC{$S <: T$} \AxiomC{$T <: S$}
\BinaryInfC{$\pamap[S] <: \pamap[T]$}
\end{prooftree}
\end{comment}
\end{definition}

\begin{lemma} Every array is also a \pamap, \palst{} or \paobj. I.e
\begin{align*}
\forall a : \parray \Rightarrow (\exists \tau.\; a : \palst \vee a : \pamap) \vee a : \paobj
\end{align*}
\end{lemma}
\begin{proof}
Let $a : \parray$ and assume that there exists some $\tau$ such that. $vht[\tau](a)$ holds. \\\\
If $listKeys(a)$ hold, then $isList[\tau](a)$. Therefor it must hold, by \ref{eq:defLstLst}, that $a : \palst$. \\\\
Else if $listKeys(a)$ does not hold, then $isMap[\tau](a)$ holds. From this and \ref{eq:defMapMap} it follows that $a : \pamap$.\\\\
If there does not exists some $\tau$ such that $vht[\tau](a)$ then it holds that $\forall \tau.\; \neg vht[\tau](a)$. This implies that $isObject(a)$ holds, from which and \ref{eq:defObjObj} follows that $a : \paobj$.

\end{proof}


\begin{lemma} Every array is either an \pamap, \palst{} or \paobj 
\begin{align}
\forall a,\tau : \palst &\Rightarrow \forall \tau_2. \; \neg (x :\pamap[\tau_2]) \wedge \neg (x: \paobj)\label{eq:onlyList}\\
\forall a,\tau : \pamap &\Rightarrow \forall \tau_2. \; \neg (x :\palst[\tau_2]) \wedge \neg (x: \paobj)\label{eq:onlyMap}\\
\forall a : \paobj &\Rightarrow \forall \tau_2. \; \neg (x :\palst[\tau_2]) \wedge \neg (x : \pamap[\tau_2])\label{eq:onlyObj}
\end{align}
\end{lemma}
\begin{proof}
The lemma is proven one equation at a time. 

\begin{description}
\item[Equation \ref{eq:onlyList}] 
\begin{subproof}
Let $a : \palst$ for some $\tau$, then \ref{eq:defLstLst} gives us that $isList[\tau](a)$ i.e. $vht[\tau](a)$ and $listKeys(a)$. \\
Assume that $\exists \tau'.\;a : \pamap[\tau']$ this, by \ref{eq:defMapMap}, implies that $\neg listKeys(a)$ \lightning. By contradiction it must hold that $\forall \tau'.\; \neg(a : \pamap[\tau'])$.\\
Assume that $a : \paobj$ this, by \ref{eq:defObjObj}, implies that $\forall \tau'.\; \neg vht[\tau'](a)$ \lightning. By contradiction it must hold $\neg (a : \paobj)$
\end{subproof}

\item[Equation \ref{eq:onlyMap}]
\begin{subproof}
Let $a : \pamap$ for some $\tau$, then \ref{eq:defMapMap} gives us that $isMap[\tau](a)$ i.e. $vht[\tau](a)$ and $\neg listKeys(a)$. \\
Assume that $\exists \tau'.\;a : \palst[\tau']$ this, by \ref{eq:defLstLst}, implies that $listKeys(a)$ \lightning. By contradiction it must hold that $\forall \tau'.\; \neg(a : \palst[\tau'])$.\\
Assume that $a : \paobj$ this, by \ref{eq:defObjObj}, implies that $\forall \tau'.\; \neg vht[\tau'](a)$ \lightning. By contradiction it must hold $\forall \neg (a : \paobj)$
\end{subproof}

\item[Equation \ref{eq:onlyObj}]
\begin{subproof}
Let $a : \paobj$ then \ref{eq:defObjObj} gives us that $isObject(a)$ i.e. $\forall \tau. \neg vht[\tau](a)$. \\
Assume that $\exists \tau'.\;a : \palst[\tau']$ this, by \ref{eq:defLstLst}, implies that $vht[\tau'](a)$ \lightning. By contradiction it must hold that $\forall \tau'.\; \neg(a : \palst[\tau'])$.\\
Assume that $\exists \tau'.\;a : \pamap[\tau']$ this, by \ref{eq:defMapMap}, implies that $vht[\tau'](a)$ \lightning. By contradiction it must hold that $\forall \tau'.\; \neg(a : \pamap[\tau'])$.\\
\end{subproof}

\end{description}
\end{proof}

\todo{Extend with rules for callable. Strings and arrays can be callable and so are anonymous functions}

\section{Coersion}

Array keys can be of type \pstring{} or \pint. 












