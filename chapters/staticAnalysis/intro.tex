Based on the findings in chapter \ref{ch:study}, a \emph{dataflow analysis} is designed and described in this chapter. The aim of the analysis is to introduce two new array types, array-lists and array-maps respectively, and subsequently report any suspicious behaviour on these arrays. Examples of suspicious behaviour could be appending to maps, using keys of type string to access lists and adding elements of a different type to an array.

The definitions of maps and lists from chapter \ref{ch:study} are redefined as:

\begin{definition}
Let $a$ be an array with exclusively integer keys, then $a$ is considered an array-list.
\label{def:newList}
\end{definition}

\begin{definition}
Let $a$ be an array with non-exclusively integer keys, then $a$ is considered an array-map.
\label{def:newMap}
\end{definition}

While running the analysis on the corpus of the dynamic analysis would be ideal, limited time forces the analysis to be restricted to a subset of PHP, introduced in section \ref{sec:langsubset}. The analysis is performed using the monotone framework. Using a control-flow graph (introduced in section \ref{sec:cfg}), a model of program states (introduced in section \ref{sec:lattice}), and finally transfer functions for each CFG-node (introduced in section \ref{sec:transferf}), an instance of the monotone framework is derived in section \ref{sec:framework}. Since PHP relies heavily on coercion, a separate section (section \ref{sec:coercion}) covers coercion of abstract values. Section \ref{sec:abseval} covers abstract evaluation and in section \ref{sec:worklist} the implementation of the analysis is briefly introduced.
