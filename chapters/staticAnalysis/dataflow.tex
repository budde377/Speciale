Based on the findings in chapter \ref{ch:study} a dataflow analysis is designed and described in this chapter. The aim of the analysis is to introduce two new array types, array-lists and array-maps respectively, and subsequently report any suspicious behaviour on these arrays. Examples of suspicious behaviour could be appending on maps, using keys of type string on lists, adding elements of a different type to an array, etc. \todo{define lists and maps, why no objects?}

While running the analysis on the corpus of the dynamic analysis would be ideal, limited time forces the analysis to be defined on a subset of PHP, introduced in section \ref{sec:langsubset}. The analysis is preformed using the monotone framework. An instance of this framework is derived, in section \ref{sec:framework}, from a control-flow-graph, introduced in section \ref{sec:cfg}, a model of program states, introduced in section \ref{sec:lattice}, and finally transfer functions for each CFG-node, introduced in section \ref{sec:transferf}. Since PHP rely heavily on coercion, a seperate section (sec. \ref{sec:coercion}) covers coercion of abstract values. Section \ref{sec:abseval} covers abstract evaluation and the implementation of the analysis is briefly introduced in section \ref{sec:worklist}.
