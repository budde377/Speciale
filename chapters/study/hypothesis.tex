\section{Hypothesis}
\label{sec:dynhypothesis}
We hypothesize that any given array throughout its lifespan from initialization to last usage can be viewed as one, and only one, of the above mentioned types, i.e. either as a list, map or an object. It is also expected that the arrays in general are acyclic and that that \emph{append}, \textit{push}, \textit{pop}, \textit{shift}, and \textit{unshift} operations are only used on arrays of type list.

If the hypothesis holds it should be possible to statically analyse the code to identify these types and detect errors related to misuse of the arrays e.g. using maps as lists or vice versa. The hypothesis is tested against a corpus consisting of ten widely used open source frameworks, by performing a dynamic analysis of the code.

The frameworks chosen all implement some kind of test suite written in PHPUnit\footnote{\url{https://phpunit.de/}}, a unit testing framework for PHP programs similar to JUnit for Java programs. By running the test suites on a modified PHP interpreter\footnote{https://github.com/Silwing/php-src}, we are able log and later analyze the structure and usage of the arrays. By using test suites instead of manually inspecting the frameworks through e.g. a browser, the aim is to gain a higher code coverage. This follows from the assumption that the developers are using code coverage as a metric of the quality of the test-suite. The corpus consists of the following open source frameworks:

\begin{itemize}
    \item \emph{WordPress}: A blogging system and a content management system\citeB{wordpress}.
    \item \emph{phpMyAdmin}: An administration panel for managing MySQL database\citeB{phpmyadmin}.
    \item \emph{MediaWiki}: The framework for creating wiki sites\citeB{mediawiki}.
    \item \emph{Joomla}: A content management system \citeB{joomla}.
    \item \emph{CodeIgniter}: A lightweight framework for building web applications\citeB{codeigniter}.
    \item \emph{phpBB}: A forum platform\citeB{phpbb}.
    \item \emph{Symfony 2}: A framework used in many major systems, such as phpBB, Magento and Drupal\citeB{symfony2}.
    \item \emph{Magento 2}: An e-commerce platform\citeB{magento2}.
    \item \emph{Zend Framework}: A framework for web development focused on simplicity, reusability and performance\citeB{zendframework}.
    \item \emph{Part}: A lightweight content management system developed by one of the authors of this thesis\citeB{part}.
\end{itemize}