This section contains the implementation details of the dynamic analysis. The last part of the section discusses flaws and limitations arisen from the chosen implementation of the analysis. The analysis consists of two phases: \textit{test suite execution} with logging of feature usage and \textit{analysis} of the logged data. The test suites are executed on a modified version of the official PHP interpreter to enable logging of feature usage.

\subsection{Logging of feature usage}
All usages of array library functions (\texttt{array\_push()}, \texttt{array\_pop()}, \texttt{array\_search()}, \texttt{count()}, etc.), array reads, array writes, assignments, and every array initialization are logged while executing the test suites. These are logged in a CSV file where each line is a log entry containing information separated by the tab character. All entries begin with  \emph{line type}, which identifies the type of the entry. The possible line types are described in the list below.

\begin{itemize}
\item \emph{array function}: Every call to a library array function\footnote{\url{http://php.net/manual/en/ref.array.php}}, such as \texttt{array\_push()} or \texttt{count()} are logged with the function name as line type. The array functions do not include the \texttt{array()} used to initialize an array, since it is a language construct and not an actual function. Some subroutines are also logged, e.g. \texttt{array\_mr\_part} used by the \texttt{array\_merge} function.
\item \texttt{array\_read}: Every read from an array is logged with the array being read from and the key used as well as the type of the value being.
\item \texttt{array\_write}: All array writes of the form \texttt{\$x[\$key] = \$y} are logged with the array being read from (\texttt{\$x}) the key (\texttt{\$key}). 
\item \texttt{array\_append}: When elements are added to the array using the append method, \texttt{\$x[] = \$y}, this is logged with the array being read from, \texttt{\$x}.
\item \texttt{assign\_*}: Every assignment is logged as either \texttt{assign\_const}, \texttt{assign\_tmp}, \texttt{assign\_var} or \texttt{assign\_ref} depending on whether the value being assigned is a constant, temporary variable, variable or reference respectively.\\
The assignment \texttt{\$x = (string) \$y} is an example of an assignment from a temporary variable. Here the the \texttt{\$y} variable is casted and saved in a temporary variable which then is assigned to \texttt{\$x}. One of these lines always follow \texttt{array\_write} and \texttt{array\_append} and is used to determine the type of value written in those lines.
\item \texttt{array\_init}: When an array is initialized in a function without the static keyword, using either the \texttt{array('key' => 'value')} construct or the corresponding bracket notation, \texttt{['key' => 'value']}, it is logged with the array being created. If the array is initialized in any other way, e.g. as a field or by array write it is not logged with \texttt{array\_init}.
\item \texttt{hash\_init}: Whenever a hash table, the underlying structure of arrays, is initialized, this is logged with the memory address of the table.
\end{itemize}

Arrays are logged as a tuple with four entries, $(t,d,s,a)$, where $t\in T\times C$ is the type of the array, $d$ is the depth of the array, $s$ is the size of the array, and $a$ is the memory address. Here $T = \{\text{List, Map, Sparse List}\}$ and $C = \{\text{Cyclic, Acyclic}\}$. The array type logged indicates the type of keys present in the array as well as whether it contains any self-references. Lists contain sequential integer keys starting from 0, Sparse Lists are any other arrays with only integer keys and Map is all arrays with at least one key of type string. Any array with a self-reference is Cyclic and all other arrays are Acyclic.

Objects are logged as their instance name, integers and floats are logged with their value, string are logged only as \texttt{string}, booleans as $0$ if \texttt{false} otherwise 1, and the null value logged as \texttt{NULL}. Strings are is generally not logged with a value, because doing so increases the file-size drastically, tends to corrupt the file when containing binary data, and has not proven necessary for the analysis.

All entries besides the \texttt{hash\_init} entries contain a line number and a file path to where the action occurred. 

\subsection{Identification of arrays}
In order to analyze how arrays are used throughout a log file, some method for identifying which arrays are mentioned on a given line is needed. E.g. in the output \ref{lst:id_array_code_out} all lines are concerning the same array with memory address \texttt{0x539}. Here identifying these arrays as the same, is a matter of checking the address. Due to the size of the test suites, relying only on the address is not a stable approach. Over time addresses will be reused and false identifications will happen. This issue can be solved by depending on the \texttt{hash\_init} line, which indicates that the a new array is initialized at some address. If such a line occurs between two usages of an address they can not necessarily be considered representing the same array. 

Since the log files are generated from running a test-suite, relying on addresses for identification alone, might result in a skewed analysis with an over-representation of arrays occurring in \emph{critical} code. Determining array equality based on initialization location in the file, should provide a more equal representation of arrays, not letting some arrays dominate the statistics. Locational identification would however identify four different arrays in example \ref{lst:id_array_code_out}, which does not reflect the program \ref{lst:id_array_code} and thus the address is still needed to identify the same array across multiple code locations.

\begin{figure}
\centering
\begin{subfigure}{\textwidth}
\begin{lstlisting}[mathescape, deletekeywords={array},basicstyle=\tiny]
hash_init       0x539
array_init      1       a.php       0       1       0       0x539
assign_tmp      1       a.php       NULL    array   1       1       3       0x539
array_append    2       a.php       1       1       3       0x539  long    4
\end{lstlisting}
\caption{Log from running file \texttt{a.php}}
\label{lst:id_array_code_out}
\end{subfigure}
\begin{subfigure}{\textwidth}
\begin{lstlisting}
$a = [1,2,3];
$a[] = 4;
\end{lstlisting}
\caption{File: \texttt{a.php}}
\label{lst:id_array_code}
\end{subfigure}
\caption{Example of the result from a run with the modified interpreter.}
\end{figure}
\begin{definition}
Given two lines, $l_1$ and $l_2$, from a log file, $R$, as described above, each containing an array $x_1\in l_1$ and $x_2\in l_2$, where $x_1 = (t_1,d_1,s_1,a_1)$ and $x_2 = (t_2,d_2,s_2,a_2)$. The arrays are said to be positional equal, $x_1\stackrel{pos}{=} x_2$, iff the two lines share the same line number, line type, and file or $a_1 = a_2$ and there is no 
\begin{align*}
\texttt{hash\_init}\;a_1
\end{align*}
line between $l_1$ and $l_2$.
\end{definition}
This definition utilizes file position and addresses in order to identify arrays. The line type has been added in order to heighten precision, since multiple different operations, on different arrays, may occur on the same line. 
\begin{definition}
Given two lines $l_1, l_2\in R$ and two arrays, $x_1\in l_1$ and $x_2 \in l_2$, then $id$ is a ID-function iff.
\begin{align*}
    id(x_1) = id(x_2) \Leftrightarrow x_1 \stackrel{pos}{=} x_2
\end{align*}
\end{definition}

When iterating through a log file, from top to bottom, IDs can be generated by keeping a mapping from locations to IDs and from addresses to IDs, and by \emph{forgetting} addresses when a \texttt{hash\_init} line is observed.

\subsection{Determining type}
Determining the type of every positional distinct array is done by first determining the key type, $t\in T$, then determining the type of the values, and from this deduce the type as either List, Map or Object as defined in the beginning of this chapter. Since an array can change type during a program, the type of the keys is more accurately a set of types. If an array is observed with different key types throughout the analysis, then the key type of the array is the set of these types. E.g, let an array be observed at one point with type List and later with type Sparse List, then the type of the array is $\{$List, Sparse List$\}$.

Detecting the key type for each array is done by traversing the file from top to bottom inspecting each line. If a line contains an array $a$ the type of the line is associated with the corresponding ID of the array.

Detecting the type of values is also done by traversing through the log file from top to bottom. Here the reads and writes from and to the arrays are used to determine the types of the values. This is done by associating all the types of the values read/written with the respective array. 

For every array with key and value type information, it is now possible to determine whether it is a list, map, object or uncategorizable. Given an array, $a$, with type information, if $a$ has multiple key types, it is considered uncategorizable. Else if $a$ contains values of a single type, it is either a map or a list, depending on the key type. If the values have multiple types, then $a$ is an object.

When the types of the arrays are determined, we can analyse the operations used with each type of array. This is done by once again iterating through the log file and associating line types with the type of the arrays.

\subsection{Compiling and running PHP with logging}
For the purpose of the dynamic analysis, Vagrant\footnote{\url{http://vagrantup.com/}} is used to create a clean and reproducible environment. A Vagrant initialization file\footnote{\url{http://github.com/Silwing/tapas-survey}} is used to setup a virtual machine running a 64-bit Ubuntu 14.04, install the necessary dependencies and compile the modified PHP Interpreter. The environment for running the corpus test suites is then ready and can be accessed via SSH on the virtual machine. The folder \texttt{/vagrant/corpus} contains a Makefile which can be used to fetch dependencies and corpus frameworks as well as running all the test suites.

All modifications to the interpreter are guarded by an ini-directive\citeB[Chapter~14.12]{programmingphp} that is disabled by default when compiling and running the modified interpreter source. The logging can be enabled via \texttt{php.ini} or for single runs as seen in figure \ref{fig:iniDirective}

\begin{figure}[ht]
\centering
\begin{subfigure}{\linewidth}
\begin{lstlisting}[language=bash]
$ php -d rb.enable_debug=1 -d rb.enable_debug_file=<path-to-log-file> <path-to-php-file>
\end{lstlisting}
\caption{Enable logging for running a single PHP file.}
\end{subfigure}
\begin{subfigure}{\linewidth}
\begin{lstlisting}
rb.enable_debug=1
rb.enable_debug_file="/path/to/output/csv/file"
\end{lstlisting}
\caption{Enable logging in php.ini.}
\end{subfigure}
\caption{How-to enable logging}
\label{fig:iniDirective}
\end{figure}

\subsection{Discussion}
Since the analysis is performed by a modified interpreter, there are some imposed limitations on the achievable precision. 

\begin{itemize}
\item  \texttt{array\_init} does not capture the type of the array at initialization. This implies that the type of arrays initialized without being assigned or manipulated afterwards are not captured by the analysis. In figure \ref{lst:dis_code} line \ref{line:print}, the call to \texttt{print\_r} does yield an \texttt{array\_init} line in the log \ref{lst:dis_code_out} but with no type, depth 1 and size 0. The size should be 3 and the type List.

\item Detecting the type relies on the type of the values read or written to the arrays. This implies that there is no reasoning about entries or arrays never read or written. This might produce some false positives.

\item Callables can be written as anonymous functions, strings, or arrays, containing an instance and a string. This analysis will fail to classify arrays containing callables, initialized differently, correctly, thus introducing false negatives. E.g in example \ref{lst:callables}; \texttt{\$callable1}, \texttt{\$callable3} and \texttt{\$callable4} are all valid callables, however \texttt{\$callable2} is not, since \texttt{\$a->f2()} is a private function.

\begin{program}
\centering
\begin{lstlisting}
class A{

    public function f1(){
        ...
    }

    private function f2(){
        ...
    }

}

function f(){
    ...
}

$a = new A();

$callable1 = [$a, "f1"];
$callable2 = [$a, "f2"];
$callable3 = "f";
$callable4 = function() use ($a){
    ...
};
\end{lstlisting}
\caption{Callables in PHP}
\label{lst:callables}
\end{program}

\item Multiple operations of the same type on different arrays leads to sharing of IDs. Following the limited information available to distinguish operations, operations such as assign to an multidimensional array leads to sharing it between the array and its sub-arrays, (see figure \ref{lst:dis_code} line \ref{line:assign}). This follows from the value first being assigned to the sub-array which then is assigned to the super-array. This leads to false negatives and could be solved if the interpreter kept character location information in addition to line number and file name.   

\end{itemize}



\begin{figure}[ht]
\centering
\begin{subfigure}{\textwidth}
\begin{lstlisting}[mathescape, deletekeywords={array},basicstyle=\tiny]
hash_init       0x2A
array_init      3       b.php     0       1       0       0x2A
hash_init       0x2B
array_write     5       b.php     0       1       0       0x2B  long    1     NULL
hash_init       0x2C
array_write     5       b.php     0       1       0       0x2C  long    2     NULL
assign_const    5       b.php     NULL    long    3
\end{lstlisting}
\caption{Log from running file \texttt{b.php}}
\label{lst:dis_code_out}
\end{subfigure}
\begin{subfigure}{\textwidth}
\begin{lstlisting}
<?php

print_r([1,2,3]); %*\label{line:print}*)

$a[1][2] = 3; %*\label{line:assign}*)

\end{lstlisting}
\caption{File: \texttt{b.php}}
\label{lst:dis_code}
\end{subfigure}
\caption{A problematic program}
\label{lst:dis}
\end{figure}