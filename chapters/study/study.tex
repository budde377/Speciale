The possibilities of the PHP array data structure allows for many different kinds of use ranging from lists over maps to trees and tables. The purpose of the dynamic analysis in this chapter is to detect patterns of array-usage in order to be able to identify a few more restrictive data types contained within the way PHP arrays are used. The results of the analysis are used to choose suitable abstraction for the static analysis in chapter \ref{ch:analysis}. The first section of this chapter describes basic definitions used in the dynamic analysis to detect patterns. With the definitions in place a hypothesis for the analysis is formed in section \ref{sec:dynhypothesis} followed by a discussion of implementation details in section \ref{sec:studyImplementation}. Section \ref{sec:analysisResults} presents the results of the dynamic analysis and section \ref{sec:studyConclusion} draws the conclusion of the analysis.

\begin{definition}
Let $a$ be an array containing values of the same type, where all integer keys from 0 to $count(a)-1$ exists. Then $a$ can be considered an array of type list.
\end{definition}

An example of an array used as a list can be seen as figure \ref{lst:list_array}, where an element is appended to the list and shifted off the beginning of the list. The values of the \texttt{\$numbers} array all share the same type, integers, and the keys, though never directly manipulated, are, at initialization, from 0 to 2. The following operations all preserve the type consensus of the values and the type of the keys. 

\begin{program}[ht]
\begin{lstlisting}
$numbers = [1,2,3];
$numbers[] = 4; // $numbers = [1,2,3,4]
$first = array_shift($numbers); // $numbers = [2,3,4]
\end{lstlisting}
\caption{Array used as a list}
\label{lst:list_array}
\end{program}

Besides the \texttt{array\_shift} function and the array append operation, $v[]$, the PHP library contains many other library functions for manipulating lists. E.g. \texttt{array\_push}, \texttt{array\_pop}, \texttt{sort}, etc.

Arrays can also explicitly define keys, which can be either a string or an integer. 

\begin{definition}
Let $a$ be an array containing values of the same type, where all integer keys from 0 to $count(a)-1$ does not exists. Then $a$ can be considered an array of type map.
\end{definition}

As the name suggests, maps can be used as a mapping from a string/integer to a value. In figure \ref{lst:map_array} the \texttt{\$text\_to\_int} array is a mapping from strings containing some numbers to its corresponding integer representation. 

\begin{program}[ht]
\begin{lstlisting}
$text_to_int = 
    [ 
        'one' => 1,
        'two' => 2,
        'three' =>  3
    ];
echo $text_to_int[$input]; 
$keys = array_keys($text_to_int);
    // $keys = ["one", "two", "three"]
$values = array_values($text_to_int);
    // $values = [1,2,3]
\end{lstlisting}
\caption{Array used as a map}
\label{lst:map_array}
\end{program}

If given an map-array, the values or keys can be fetched using the functions \texttt{array\_values} or \texttt{array\_keys} respectively. These functions returns an array of list type. 

Finally arrays can be treated as objects, i.e. the entries can be viewed as properties of arbitrary type. These arrays could be replaced by the \texttt{stdClass} which mainly is used for its dynamic properties, just like arrays. Some of the build in arrays of PHP can be considered objects, this include the \texttt{\$\_SERVER} array\footnote{\url{http://php.net/manual/en/reserved.variables.server.php}}. This array contains server and execution environment information, which are of different types, e.g. \texttt{\$\_SERVER['argv']} is an array of arguments passed to the interpreter, and \texttt{\$\_SERVER['REQUEST\_TIME']} is an integer UNIX timestamp of the start of the request. 

\begin{definition}
Let $a$ be an array containing values of different type. Then $a$ can be considered an array of type object.
\end{definition}


\section{Hypothesis}
\label{sec:dynhypothesis}
Our hypothesis is that any given array throughout its lifespan, from initialization to last usage, can be viewed as one, and only one, of the above mentioned types. I.e. either as a list, a map or an object. It is also expected that the arrays in general are acyclic and that that append, push, pop, shift and unshift operations are only used on arrays of type list.

If the hypothesis holds it should be possible to statically analyse the code to identify these types and detect errors related to misuse of the arrays e.g. using maps as lists or vice versa. The hypothesis is tested against a corpus consisting of ten widely used open source frameworks, by performing a dynamic analysis of the code.

The frameworks chosen all implement some kind of test suite written in PHPUnit\footnote{\url{https://phpunit.de/}}. A unit testing framework for PHP progrms similar to JUnit for Java programs. By running the test suites on a modified PHP interpreter\footnote{https://github.com/Silwing/php-src}, we are able log and later analyze the structure and usage of the arrays. By using test suites instead of manually inspecting the frameworks through e.g. a browser, the aim is to gain a higher code coverage. This follows from the assumption that the developers are using code coverage as a metric of the quality of the test-suite. The corpus consists of the following open source frameworks:

\begin{itemize}
    \item \emph{WordPress}: A blogging system and a content management system\citeB{wordpress}.
    \item \emph{phpMyAdmin}: An administration panel for managing MySQL database\citeB{phpmyadmin}.
    \item \emph{MediaWiki}: The framework for creating wiki sites\citeB{mediawiki}.
    \item \emph{Joomla}: A content management system \citeB{joomla}.
    \item \emph{CodeIgniter}: A lightweight framework for building web applications\citeB{codeigniter}.
    \item \emph{phpBB}: A forum platform\citeB{phpbb}.
    \item \emph{Symfony 2}: A framework used in many major systems, such as phpBB, magento and Drupal\citeB{symfony2}.
    \item \emph{Magento 2}: An e-commerce platform\citeB{magento2}.
    \item \emph{Zend Framework}: A framework for web development focused on simplicity, reusability and performance\citeB{zendframework}.
    \item \emph{Part}: A lightweight content management system developed by one of the authors of this thesis\citeB{part}.
\end{itemize}


\section{Implementation}
\label{sec:studyImplementation}
\section{Implementation}
\label{sec:studyImplementation}
This section contains the implementation details of the dynamic analysis. The last part of the section discusses flaws and limitations arisen from the chosen implementation of the analysis. The analysis consists of two phases: \textit{test suite execution} with logging of feature usage, and \textit{analysis} of the data logged. The test suites are executed on a modified version of the official PHP interpreter to enable logging of feature usage.

\subsection{Logging of feature usage}
All usages of array reads, array writes, assignments, array library functions (\texttt{array\_push()}, \texttt{array\_pop()}, \texttt{array\_search()}, \texttt{count()}, etc.), and every array initialization are logged while executing the test suites. These are logged in a CSV file where each line is a log entry containing information separated by the tab character. All entries begin with  \emph{line type}, which identifies the type of the entry. The possible line types are described in the list below.

\begin{itemize}
\item \emph{array function}: Every call to a library array function \citeB{phparray}, such as \texttt{count()} or \texttt{array\_push()} is logged with the function name as line type. The array functions do not include the \texttt{array()} used to initialize an array, since it is a language construct and not an actual function. Some subroutines are also logged when dealing with multiple arrays in the same function, e.g. \texttt{array\_mr\_part} used by the \texttt{array\_merge} function.
\item \texttt{array\_read}: Every read from an array is logged with the array being read from and the key used, as well as the type of the value being read.
\item \texttt{array\_write}: All array writes of the form \texttt{\$x[\$key] = \$y} are logged with the array being read from (\texttt{\$x}) and the key (\texttt{\$key}). 
\item \texttt{array\_append}: When elements are added to the array using the append method, \texttt{\$x[] = \$y}, this is logged with the array being read from, \texttt{\$x}.
\item \texttt{assign\_*}: Every assignment is logged as either \texttt{assign\_var}, \texttt{assign\_tmp}, \texttt{assign\_const} or \texttt{assign\_ref} depending on whether the value being assigned is a constant, temporary variable, variable or reference respectively.\\
The assignment \texttt{\$x = (string) \$y} is an example of an assignment from a temporary variable. Here the \texttt{\$y} variable is cast and saved in a temporary variable which is then assigned to \texttt{\$x}. One of these lines always follow \texttt{array\_write} and \texttt{array\_append} and is used to determine the type of value written in those lines.
\item \texttt{array\_init}: When an array is initialized without the static keyword, using either the \texttt{array('key' => 'value')} construct or the corresponding bracket notation, \texttt{['key' => 'value']}, it is logged with the array being created. If the array is initialized in any other way, e.g.\ as a field or by array write, it is not logged with \texttt{array\_init}.
\item \texttt{hash\_init}: Whenever a hash table, the underlying structure of arrays, is initialized, this is logged with the memory address of the table.
\end{itemize}

Arrays are logged as a tuple with four entries $(t,d,s,a)$, where $t\in T\times C$ is the type of the array, $d$ is the depth of the array, $s$ is the size of the array, and $a$ is the memory address. Here $T = \{\text{List, Map, Sparse List}\}$ and $C = \{\text{Cyclic, Acyclic}\}$. The array-type logged indicates the type of keys present in the array, see definitions \ref{def:list} and \ref{def:map}, as well as whether it contains any self-references. Any array with a self-reference is cyclic and all other arrays are acyclic.

Objects are logged with their class name, integers and floats are logged with their value, strings are logged only as \texttt{string}, booleans as $0$ if \texttt{false} otherwise 1, and the null value logged as \texttt{NULL}. Strings are generally not logged with a value, because doing so increases the file size drastically, tends to corrupt the file when containing binary data, and has not proven to be useful for the analysis.

All entries besides the \texttt{hash\_init} entries contain a line number and a file path to where the action occurred. 

\subsection{Identification of arrays}
In order to analyse how arrays are used throughout an execution of a test suite, some method for identifying which arrays that are mentioned on a given line is needed. For instance, in the output depicted in figure \ref{lst:id_array_code_out} all lines concern the same array with memory address \texttt{0x539}. Here, identifying these arrays as the same is a matter of checking the address. Due to the size of the test suites, relying only on the address is not a sound approach. Over time addresses will be reused and false identifications will happen. This issue can be solved by depending on the \texttt{hash\_init} line, which indicates that a new array is initialized at some address. If such a line occurs between two usages of an address they can not be considered representing the same array. 

\begin{figure}
\centering
\begin{subfigure}{\textwidth}
\begin{lstlisting}[mathescape, deletekeywords={array},basicstyle=\tiny]
hash_init       0x539
array_init      1       a.php       0       1       0       0x539
assign_tmp      1       a.php       NULL    array   1       1       3       0x539
array_append    2       a.php       1       1       3       0x539  long    4
\end{lstlisting}
\caption{Log from running file \texttt{a.php}}
\label{lst:id_array_code_out}
\end{subfigure}
\begin{subfigure}{\textwidth}
\begin{lstlisting}
$a = [1,2,3];
$a[] = 4;
\end{lstlisting}
\caption{File: \texttt{a.php}}
\label{lst:id_array_code}
\end{subfigure}
\caption{Example of the result from a run with the modified interpreter.}
\end{figure}

Since the log files are generated from running a test suite, relying on addresses for identification alone might result in a skewed analysis with an over-representation of arrays occurring in \emph{critical} code. Determining array equality based on initialization location in the file should provide a more equal representation of arrays, not letting some arrays dominate the statistics. Locational identification would, however, identify four different arrays in example \ref{lst:id_array_code_out}, which does not reflect the program \ref{lst:id_array_code} and thus the address is still needed to identify the same array across multiple uses on different code locations.


\begin{definition}
Given two lines, $l_1$ and $l_2$, from a log file $R$, formatted as described above, each containing an array, $x_1\in l_1$ and $x_2\in l_2$, where $x_1 = (t_1,d_1,s_1,a_1)$ and $x_2 = (t_2,d_2,s_2,a_2)$, the arrays are said to be positionally equal $x_1\stackrel{pos}{=} x_2$, if and only if the two lines share the same line number, line type, and file or $a_1 = a_2$ and there is no 
\begin{align*}
\texttt{hash\_init}\;a_1
\end{align*}
line between $l_1$ and $l_2$.
\end{definition}

This definition utilizes file position and addresses in order to identify arrays. The line type has been added in order to heighten precision, since multiple different operations, on different arrays, may occur on the same line. 

\begin{definition}
Given two lines $l_1, l_2\in R$ and two arrays, $x_1\in l_1$ and $x_2 \in l_2$, then $id$ is an ID-function if and only if
\begin{align*}
    id(x_1) = id(x_2) \Leftrightarrow x_1 \stackrel{pos}{=} x_2
\end{align*}
\end{definition}

When iterating through a log file from top to bottom, IDs can be generated by keeping a mapping from locations to IDs and from addresses to IDs, and by \emph{forgetting} addresses when a \texttt{hash\_init} line is observed.

\subsection{Determining type}
Determining the type of every positionally distinct array is done by first determining the key type, $t\in T$, then determining the type of the values, and from this deduce the type as either list, map or object as defined in the beginning of this chapter. Since we want to test the hypothesis of whether an array-type changes, the type of the keys is a set of types. If an array is observed with different key types throughout the analysis, then the key type of the array is the set of these types. E.g, let an array be observed at one point with type list and later with type sparse list, then the type of the array is $\{$List, Sparse List$\}$.

Detecting the key type for each array is done by traversing the file from top to bottom inspecting each line. If a line contains an array $a$ the type of the line is associated with the corresponding ID of the array.

Detecting the type of values is also done by traversing through the log file from top to bottom. Here the reads and writes from and to the arrays are used to determine the types of the values. This is done by associating all the types of the values read/written with the respective array. 

For every array with key and value type information, it is now possible to determine whether it is a list, map, object or uncategorizable. Given an array, $a$, with type information, if $a$ has multiple key types, it is considered uncategorizable. Otherwise if $a$ contains values of a single type, it is either a map or a list, depending on the key type. If the values have multiple types, then $a$ is an object.

When the types of the arrays are determined, we can analyse the operations used with each type of array. This is done by once again iterating through the log file and associating line types with the types of the arrays.

\subsection{Compiling and running PHP with logging}
The source code needed to compile and run the modified PHP interpreter can be found at \url{http://github.com/Silwing/tapas}. For the purpose of the dynamic analysis, Vagrant\citeB{vagrantup} is used to create a clean and reproducible environment. A Vagrant initialization file is used to setup a virtual machine running a 64-bit Ubuntu 14.04, install the necessary dependencies and compile the modified PHP Interpreter. The environment for running the corpus test suites is then ready and can be accessed via SSH on the virtual machine. The folder \texttt{/vagrant/corpus} contains a Makefile which can be used to fetch dependencies and corpus frameworks as well as running all the test suites.

All modifications to the interpreter are guarded by an ini-directive\citeB[Chapter~14.12]{programmingphp} that is disabled by default when compiling and running the modified interpreter source. The logging can be enabled via \texttt{php.ini} or for single runs as seen in figure \ref{fig:iniDirective}

\begin{figure}[ht]
\centering
\begin{subfigure}{\linewidth}
\begin{lstlisting}[language=bash]
$ php -d rb.enable_debug=1 -d rb.enable_debug_file=<path-to-log-file> <path-to-php-file>
\end{lstlisting}
\caption{Enable logging for running a single PHP file.}
\end{subfigure}
\begin{subfigure}{\linewidth}
\begin{lstlisting}
rb.enable_debug=1
rb.enable_debug_file="/path/to/output/csv/file"
\end{lstlisting}
\caption{Enable logging in php.ini.}
\end{subfigure}
\caption{How to enable logging}
\label{fig:iniDirective}
\end{figure}

\subsection{Limitations}
\label{sec:limitations}
Since the analysis is performed by a modified interpreter, there are some imposed limitations on the achievable precision. 

\begin{itemize}
\item  \texttt{array\_init} does not capture the type of the array at initialization. This implies that the types of arrays initialized without being assigned or manipulated afterwards are not captured by the analysis. In figure \ref{lst:dis_code} line \ref{line:print}, the call to \texttt{print\_r} does yield an \texttt{array\_init} line in the log \ref{lst:dis_code_out} but with no type, depth 1 and size 0. The size should be 3 and the type should be list.

\item Detecting the type of arrays relies on the type of the values read from or written to the arrays. This implies that there is no reasoning about arrays never being read or written. Some of these arrays may fall into the uncategorizable type but remain undetected by the analysis.

\item Callables can be written as anonymous functions, strings, or arrays containing either an instance or a string representing a class name together with string representing a function name. Arrays of callables should be considered lists, however this analysis will classify them as objects. Program \ref{lst:callables} shows different types of callables.

\begin{program}
\centering
\begin{lstlisting}
class A{

    public function f1(){
        ...
    }

}

function f(){
    ...
}

$a = new A();

$callable1 = [$a, "f1"];
$callable2 = ["A", "f1"];
$callable3 = "f";
$callable4 = function() use ($a){
    ...
};
\end{lstlisting}
\caption{Callables in PHP}
\label{lst:callables}
\end{program}

\item Multiple operations of with the same type using different arrays lead to sharing of IDs. Following the limited information available to distinguish between operations, operations such as assign to an multidimensional array leads to sharing the ID between the array and its sub-arrays, (see figure \ref{lst:dis_code} line \ref{line:assign}). This follows from the value first being assigned to the sub-array, which is then assigned to the super-array. Combining multiple arrays may lead to uncategorizable arrays even if each array is categorizable. A possible solution would be if the interpreter kept character location information in addition to line number and file name.   

\end{itemize}



\begin{figure}[ht]
\centering
\begin{subfigure}{\textwidth}
\begin{lstlisting}[mathescape, deletekeywords={array},basicstyle=\tiny]
hash_init       0x2A
array_init      3       b.php     0       1       0       0x2A
hash_init       0x2B
array_write     5       b.php     0       1       0       0x2B  long    1     NULL
hash_init       0x2C
array_write     5       b.php     0       1       0       0x2C  long    2     NULL
assign_const    5       b.php     NULL    long    3
\end{lstlisting}
\caption{Log from running file \texttt{b.php}}
\label{lst:dis_code_out}
\end{subfigure}
\begin{subfigure}{\textwidth}
\begin{lstlisting}
<?php

print_r([1,2,3]); %*\label{line:print}*)

$a[1][2] = 3; %*\label{line:assign}*)

\end{lstlisting}
\caption{File: \texttt{b.php}}
\label{lst:dis_code}
\end{subfigure}
\caption{A problematic program}
\label{lst:dis}
\end{figure}


%\section{Feature usage analysis}
%\label{sec:featureAnalysis}
%In the following section we describe which features we analyze and why the dynamic analysis of these features are useful.

As the test-suite is run, the modified interpreter logs all usages of array library functions (\texttt{array\_push()}, \texttt{array\_pop()}, \texttt{array\_search()}, \texttt{count()}, etc.), array reads, array writes, assignments, and every array initialization. These are logged in a CSV file where each line is a log entry containing information separated by the tab character. All entries are starting with a \emph{line type}, which identifies the type of the entry.

\begin{itemize}
\item \emph{array function}: Every call to an library array function\footnote{\url{http://php.net/manual/en/ref.array.php}}, such as \texttt{array\_push()} or \texttt{count()} are logged with the function name as line type. The array functions does not include the \texttt{array()} used to initialize an array, since it is a language construct and not an actual function. Some subroutines are also logged, e.g. \texttt{array\_mr\_part} used by the \texttt{array\_merge} function.
\item \texttt{array\_read}: Every read from an array is logged with the array being read from and the key used as well as the type of the value being.
\item \texttt{array\_write}: All array writes, on the form \texttt{\$x[\$key] = \$y}, are logged with the array being read from, \texttt{\$x}, the key, and the \texttt{\$key}. 
\item \texttt{array\_append}: When elements are added to the array using the append method, \texttt{\$x[] = \$y}, this is logged with the array being read from, \texttt{\$x}.
\item \texttt{assign\_*}: Every assignment is logged as either \texttt{assign\_const}, \texttt{assign\_tmp}, \texttt{assign\_var} or \texttt{assign\_ref} depending wheteher the value being assigned is a constant, temporary variable, variable or reference respectively.\\
The assignment \texttt{\$x = (string) \$y} is an example of an assignment from a temporary variable. Here the the \texttt{\$y} variable is casted and saved in a temporary variable which then is assigned to \texttt{\$x}. One of these lines always follow \texttt{array\_write} and \texttt{array\_append} and is used to determine the type of value written in those lines.
\item \texttt{array\_init}: When an array is initialized, in a function without the static keyword, using either the \texttt{array('key' => 'value')} construct or the corresponding bracket notation, \texttt{['key' => 'value']}, it is logged with the array being created. If the array is initialized in any other way, e.g. as a field or by array write it is not logged with \texttt{array\_init}.
\item \texttt{hash\_init}: Whenever a hash table, the underlying structure of arrays, is initialized, this is logged with the memory address of the table.
\end{itemize}

Arrays are logged as a tuple with four entries, $(t,d,s,a)$, where $t$ is the type of the array, $d$ is the depth of the array, $s$ is the size of the array, and $a$ is the memory address. The array types are the types in table \ref{tab:array_log_type}. Any given array are either a cyclic or acyclic list, sparse list, or map. Array keys are logged with their type and value. 

Objects are logged as their instance name, integers and floats are logged with their value, string are logged only as \texttt{string}, booleans as $0$ if \texttt{false} else 1, and the null value logged as \texttt{NULL}. The string is generally not logged with a value because doing so increases the file-size drastically, tend to corrupt the file when containing binary data and has not proven necessary for the analysis.

Each entry, but the \texttt{hash\_init} entries, contains a line number and a file path to where the action occurred. 

\begin{table}
\centering
\begin{tabularx}{0.8\textwidth}{@{} c|Y @{}}
{\bf Type} & {\bf Description} \\\hline\hline
0 & No information\\
1 & The array is a list. I.e. all the numbers $1\dots s$, where $s$ is the size of the array are keys in the array.\\
2 & The array is a sparse list. I.e. all keys are integers and the array is not a list.\\
4 & The array is a map. I.e. the array is not a list or a sparse list.\\
9 & The array is a cyclic list.\\
10 & The array is a cyclic sparse list.\\
12 & The array is a cyclic map.
\end{tabularx}
\caption{Array types}
\label{tab:array_log_type}
\end{table}
\subsection{Identifying arrays}
In order to analyse how an array changes throughout a run, some method for expressing that two arrays are the same, is needed. 
\begin{definition}
Given two lines, $l_1$ and $l_2$, from a log file, $R$, as described above, each containing an array $x_1\in l_1$ and $x_2\in l_2$, where $x_1 = (t_1,d_1,s_1,a_1)$ and $x_2 = (t_2,d_2,s_2,a_2)$. The arrays are said to be the same, $x_1\sim x_2$, iff the two lines share the same line number, line type, and file or $a_1 = a_2$ and there is no 
\begin{align*}
\texttt{hash\_init}\;a_1
\end{align*}
line between $l_1$ and $l_2$.
\end{definition}
From this definition, we can define an id for every observed array as follows.
\begin{definition}
Given two lines $l_1, l_2\in R$ and two arrays, $x_1\in l_1$ and $x_2 \in l_2$, then $id$ is a function from arrays to natural numbers, such that

\begin{align*}
    id(x_1) = id(x_2) \Leftrightarrow x_1 \sim x_2
\end{align*}

\end{definition}

\subsection{Determining type}
Given a log, $R$, generated from running a test suite. Determining the type of every distinct array is done by first determining the key type for each id, then determining the type of the values for every id and from this deduce the type of each id. Since an id can represent many observed arrays, the type of the keys is more accurately a set of types. E.g. if an array, with some id, is observed to have key type \emph{list} and another array, with the same id, is observed to have key type \emph{map}, then the key type of the id is $\{list, map\}$, represented as a bit-wise-or of the type integers form table \ref{tab:array_log_type}, in this case $1 | 4 = 5$.

The type of the id can be either list, map or object. Here the lists are arrays with key type list and values of the same type, maps are arrays with key type sparse list or map and values of the same type, and objects are arrays with a single key type and values of different type. Any array with multiple key type is uncategorizable and our hypothesis is that this group is small.

Detecting the key type for each id, in a given log, $R$, is done by traversing the file from top to bottom inspecting each line. If a line contains an array $x = (t,d,s,a)$, with an id $i= id(x)$, then if $x$ is the first array observed with id $i$, the id and key type, $t$ is remembered. Otherwise the there is an old type, $t_{old}$, associated with $i$. This type is replaced with $t' = t | t_{old}$.

Detecting the type of values is also done by traversing through the log file from top to bottom. Here the reads and writes from and to the arrays are used to determine the types of the values, by associating all the types of the values read/written with the respective array id. When the log has been traversed the types for each id are compared and it is determined whether the id has one or more types. 

There is no reasoning about entries never read or written which might produce some false positives. Furthermore since callables can be written as anonymous functions, strings, or arrays, containing an instance and a string, this analysis might fail to classify arrays containing callables as arrays containing elements of the same type, thus introducing false negatives.   

From the two previous analysis it is now possible to determine, for every id with key and value type information, whether it is a list, map, object or uncategorizable. Any cyclic types are treated as their acyclic counterpart.

When the types of the arrays are determined we can analyse the operations used with each type of array. Our hypothesis is that append, push, pop, shift and unshift operations are good predictors for list types of arrays. If that is the case we can detect when a map or object type array is used as a list which is likely to be an error. We can also detect when list types are accessed with string-type keys which is also likely to be an error.

\begin{comment}
Combining the information from the two previous sections it is possible to  identify three groups of arrays: lists, maps and objects. What will be interesting to see is whether identified arrays stay in one of these groups for its entire lifetime or whether they change types throughout a test-run.

If arrays almost never changes types we can make our static analysis assume that arrays never change types without introducing too many errors by being unsound.
\end{comment}
\subsection{Critique}

Apart from the sources of false positives and false negatives mentioned in the previous section, other points of critique include.
\begin{itemize}
\item \texttt{array\_init} does not capture the type of the array at initialization. This implies that the type of arrays initialized without being assigned or manipulated afterwards, e.g. as an inline argument to a library function, are not captured by the analysis.
\item Multiple operations of the same type on different arrays leads to sharing of ids. Following the limited information available to distinguish operations, operations such as assign to an multidimensional array leads to sharing it between the array and its sub-arrays since a value first is assigned to the sub-array which then is assigned to the super-array. This leads to false negatives and could be solved if the interpreter kept character location information in addition line number and file.   
\item The liberal nature of array initialization implies that an array can be initialized at two different code locations for different input even though they are semantically the same array. These arrays should be the same, thus share the same id. The analysis fails to observe this and thus might introduce false positives. 
\item An array might in every possible execution have values, or keys, of the same type, but when compared over several executions have different types. This might lead to false negatives.
\end{itemize}

\section{Results}
\label{sec:analysisResults}

Table \ref{tab:cyclic_arrays} shows that across all frameworks in the corpus less than 1\% of the arrays are detected as being cyclic. Cyclic arrays are created using the the PHP reference operator, \texttt{\&}, which must be used explicitly. Due to the explicit reference operator, cyclic arrays do not occur as an unintentional side effect of something else\todo{something else?}. The largest amount of cyclic arrays detected in the corpus is in PhpMyAdmin with a total of 10 cyclic arrays out of 3,373 identified arrays. By assuming that arrays are acyclic, the static analysis would not have to take recursive types into consideration. Since the results show that almost every framework contains some cyclic arrays, the static analysis must handle recursive types in some way. The small amount of cyclic arrays, however, does imply that an imprecise handling of recursive types should not impact the overall precision much.
\todo{Rewrite this with new findings. See comment.}
\begin{comment}
NG cyclic arrays are the cyclic arrays that are NOT the \$GLOBAL array, which is recursive. 
Notice that no creation of cyclic arrays has been found, thus the last arrays might also be the globals array.
The assumption of acyclic arrays is not totally wrong.
\end{comment}
\begin{table}[htbp]
\begin{center}
\begin{tabular}{l| r  r  r}
Framework & \# Arrays & \# Cyclic arrays & \# NG Cyclic arrays  \\ \hline \hline
Code igniter & 331 & 0 & 0 \\
Joomla & 1969 & 2 & 0\\ 
Magento2 & 6942 & 0 & 0\\ 
Mediawiki & 27368 & 1 & 0\\ 
Part & 378 & 0 & 0\\ 
phpBB & 2529 & 1 & 0\\
PhpMyAdmin & 3373 & 10 & 0\\
Symfony & 3707 & 6 & 6\\ %Ten found when running program, but only six were not errors
Wordpress & 3054 & 1 & 0\\ 
Zend Framework 2 & 4381 & 3 & 2
\end{tabular}
\end{center}
\caption{Amount of cyclic arrays detected in the corpus}
\label{tab:cyclic_arrays}
\end{table}

Figure \ref{fig:array_types} shows the distribution of array types for the frameworks.  Between 4\% and 12\% of the arrays are uncategorizable. These include false uncategorizables originating from flaws in the array identification. If multiple categorizable arrays from different categories are identified as a single array it might end up in the uncategorizable part of the distribution.

The Object group marked with List is by definition categorized as objects, but they might fit better into the List category, as lists of a top level type. \todo{elaborate}

\begin{table}[htbp]
\begin{adjustbox}{center}
\begin{tabular}{l | c c c c c c c}
    &   List    &   Map &   Sparse List &   Object  &   Object (L)   &   Object (SL)   &   Uncategorizable \\
\hline \hline
Code Igniter    &   39.66\% &   36.21\% &   2.59\%  &   12.93\% &   2.59\%  &   0.00\%  &   6.03\% \\
Joomla          &   30.78\% &   39.13\% &	2.17\%	&   20.02\% &	3.66\%	&   0.11\%  &   4.12\% \\
Magento 2	    &   23.54\% &   46.51\% &	3.38\%	&   17.93\% &	2.63\%	&   0.70\%  &   	5.30\% \\
MediaWiki	    &   32.48\%	&   32.23\% &	2.69\%	&   15.60\% &	8.20\%	&   0.49\%  &   	8.32\% \\
Part	        &   33.33\%	&   39.10\% &	0.00\%	&   12.82\% &	5.77\%	&   0.00\%  &   	8.97\% \\
phpBB	        &   27.13\%	&   33.33\% &	3.17\%	&   25.11\% &	4.33\%	&   0.14\%  &   	6.78\% \\
PhpMyAdmin	    &   33.24\%	&   33.43\% &	2.09\%	&   14.06\% &	5.89\%	&   0.38\%  &   	10.92\% \\
Symfony	        &   34.32\%	&   28.01\% &	1.99\%	&   14.86\% &	8.63\%	&   0.21\%  &   	11.99\% \\
WordPress	    &   35.50\%	&   33.03\% &	2.02\%	&   14.11\% &	6.61\%	&   0.45\%  &   	8.29\% \\
Zend Framework 2&	30.99\%	&   35.07\% &	1.08\%	&   19.78\% &	6.25\%	&   0.00\%  &   	6.82\% 
\end{tabular}
\end{adjustbox}
\caption{Distribution of different array types}
\label{tab:array_types}
\end{table}

% There might be a problem with our classification. E.g. the groups decided to be objects are prob. not objects, but really lists with different type. It might be an idea to go through the groups and analyse. The only "real" objects are maps (what is callables then? Special case?). If it is the case that the M lists are in fact S lists, then we may make the assumption that when a list is observed, it has the same type and really doesn't change over time. We need however to keep track of key -> type of maps, since they may prove to be objects rather than maps. A further study on how the depth/size changes for each group would be nice.  
\begin{figure}[htbp]
\centering
\includegraphics[width=\textwidth]{chapters/study/g1.png}
\caption{Distribution of different types of arrays}
\label{fig:array_types}
\end{figure}

Figure \ref{fig:type_operations} shows the distribution of operations on the arrays over the different array types from figure \ref{fig:array_types}. Write and append correspond to the language features for writing to arrays:

\begin{lstlisting}
$a = [];
$a[0] = 42; // array write
$a[] = 1337; // array append
\end{lstlisting}

These operations are by far the most used. The built-in push function is equivalent to the append operation if given only a single argument. The documentation recommends the append operation in such situations for performance reasons, which aligns with the use of append over push in the figure. The operation on arrays of map and object type consists almost entirely of write operations, whereas arrays of type list have some write operation but mostly append operations. This indicates that append is a good predictor for arrays in the list category.

The distribution of operations support the claim that the List-marked objects fit better into the list category than the object category, since multiple list operations are frequently used with these arrays.

\begin{figure}[htbp]
\centering
\includegraphics[width=\textwidth]{chapters/study/g2.png}
\caption{Distribution of array-changing operations over array types}
\label{fig:type_operations}
\end{figure}


\section{Conclusion}
\label{sec:studyConclusion}
The purpose of the dynamic analysis was to determine whether PHP array usage can be split into semantic categories and if arrays stay in one category during their lifetime. 

The results show that generally arrays keep the same type during their lifetime. Further more the usage of specific array operations proved almost exclusive for lists. This information can be used to define unexpected behavior reported by the static analysis.

A significant amount of arrays turned out to be objects with list-keys by the initial definition. These objects indicates that the object type is not providing significant information in itself, and shows the possibility that the definitions of maps and lists consume the object type; i.e. letting maps and lists allow values of different type.

Almost every framework in the corpus contains some cyclic arrays why the static analysis have to take the possibility of recursive types into consideration. However the small amount of cyclic arrays indicate that an imprecise approach will have a minimal impact on the overall precision.


\begin{comment}
HASHSES

/users/budde377/encasa/tapas-survey/corpus/git/Part
657bc1fed0eafac1acf6d82ce42ec2243d146207
/users/budde377/encasa/tapas-survey/corpus/git/phpmyadmin
d295244b7b9b21dab4c97e6bbcca182e98ce7d67
/users/budde377/encasa/tapas-survey/corpus/git/mediawiki
97c4e93eefc7182fb4a166e752d44e8853fe3218
/users/budde377/encasa/tapas-survey/corpus/git/joomla-cms
27b137e78925aedeebff9bf7d56be6b5ba080a4a
/users/budde377/encasa/tapas-survey/corpus/git/CodeIgniter
64d1d82e03ace010dcf03c509cf6b87e0da27ff4
/users/budde377/encasa/tapas-survey/corpus/git/phpbb
e132e9ba76e63f71a6019b79c7a42417b3b633b4
/users/budde377/encasa/tapas-survey/corpus/git/symfony
e91058089c900071362270f0756b166bf4028edc
/users/budde377/encasa/tapas-survey/corpus/git/magento2
d0131d777e173f60c2f6acd435173b5c7bb513c5
/users/budde377/encasa/tapas-survey/corpus/git/zf2
97f67e0d06f1693a0b6e9e62561ca482adc85f7c


SVN Stuff (Wordpress)

[budde377@casa01 trunk] (master) \$ svn info
Path: .
Working Copy Root Path: /encasa/budde377/tapas-survey/corpus/svn/trunk
URL: https://develop.svn.wordpress.org/trunk
Relative URL: ^/trunk
Repository Root: https://develop.svn.wordpress.org
Repository UUID: 602fd350-edb4-49c9-b593-d223f7449a82
Revision: 31753
Node Kind: directory
Schedule: normal
Last Changed Author: SergeyBiryukov
Last Changed Rev: 31753
Last Changed Date: 2015-03-12 15:56:34 +0100 (Thu, 12 Mar 2015)



\end{comment}