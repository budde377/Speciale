The PHP array data structure allows for many different kinds of use ranging from lists over maps to trees and tables. The purpose of the dynamic analysis in this chapter is to detect patterns of array-usage in order to be able to identify some more restrictive data types contained within the way PHP arrays are used. The results of the analysis are used to choose a suitable abstraction for the static analysis in chapter \ref{ch:analysis}. The first section of this chapter describes basic definitions used in the dynamic analysis to detect patterns. With the definitions in place a hypothesis for the analysis is formed in section \ref{sec:dynhypothesis} followed by a discussion of implementation details in section \ref{sec:studyImplementation}. Section \ref{sec:analysisResults} presents the results of the dynamic analysis and section \ref{sec:studyConclusion} draws the conclusion of the analysis.

\begin{definition}
\label{def:list}
Let $a$ be an array of size $n$ containing values of the same type, where all integer keys from 0 to $n-1$ exists. Then $a$ can be considered an array of type list.
\end{definition}

An example of an array used as a list can be seen in program \ref{lst:list_array}, where an element is appended to the list and shifted off the beginning of the list. The values of the \texttt{\$numbers} array all share the same type (integers) and the keys, though never directly manipulated, range at initialization from 0 to 2. The following operations all preserve the type consensus of the values and the type of the keys. 

\begin{program}[ht]
\begin{lstlisting}
$numbers = [1,2,3];
$numbers[] = 4; // $numbers = [1,2,3,4]
$first = array_shift($numbers); // $numbers = [2,3,4]
\end{lstlisting}
\caption{Array used as a list}
\label{lst:list_array}
\end{program}

Besides the \texttt{array\_shift} function and the array append operation, \texttt{\$v[]}, the PHP library contains many other library functions for manipulating lists, e.g. \texttt{array\_push}, \texttt{array\_pop}, \texttt{sort}, etc.

Arrays can also explicitly define keys, which can be either a string or an integer. 

\begin{definition}
\label{def:map}
Let $a$ be an array of size $n$ containing values of the same type, where some integer key from 0 to $n-1$ does not exist. Then $a$ can be considered an array of type map.
\end{definition}

As the name suggests, maps can be used as a mapping from a string/integer to a value. In the dynamic analysis a map with only integer keys is also denoted a sparse list. In program \ref{lst:map_array} the \texttt{\$text\_to\_int} array is a mapping from strings containing number names to their corresponding integer representation. 

\begin{program}[ht]
\begin{lstlisting}
$text_to_int = 
    [ 
        'one' => 1,
        'two' => 2,
        'three' =>  3
    ];
echo $text_to_int[$input]; 
$keys = array_keys($text_to_int);
    // $keys = ["one", "two", "three"]
$values = array_values($text_to_int);
    // $values = [1,2,3]
\end{lstlisting}
\caption{Array used as a map}
\label{lst:map_array}
\end{program}

If given a map-array, the values or keys can be fetched using the functions \texttt{array\_values} or \texttt{array\_keys} respectively. These functions return an array of the list type. 

Finally arrays can be treated as objects, i.e. the entries can be viewed as properties of arbitrary type. These arrays could be replaced by the \texttt{stdClass} which mainly is used for its dynamic properties, just like arrays. Some of the build-in arrays of PHP can be considered objects, including the \texttt{\$\_SERVER} array\footnote{\url{http://php.net/manual/en/reserved.variables.server.php}}. This array contains server and execution environment information, which are of different types, e.g. \texttt{\$\_SERVER['argv']} is an array of arguments passed to the interpreter, and \texttt{\$\_SERVER['REQUEST\_TIME']} is an integer UNIX timestamp of the start of the request. 

\begin{definition}
\label{def:object}
Let $a$ be an array containing values of different types. Then $a$ can be considered an array of type object.
\end{definition}


\begin{comment}
HASHSES

/users/budde377/encasa/tapas-survey/corpus/git/Part
657bc1fed0eafac1acf6d82ce42ec2243d146207
/users/budde377/encasa/tapas-survey/corpus/git/phpmyadmin
d295244b7b9b21dab4c97e6bbcca182e98ce7d67
/users/budde377/encasa/tapas-survey/corpus/git/mediawiki
97c4e93eefc7182fb4a166e752d44e8853fe3218
/users/budde377/encasa/tapas-survey/corpus/git/joomla-cms
27b137e78925aedeebff9bf7d56be6b5ba080a4a
/users/budde377/encasa/tapas-survey/corpus/git/CodeIgniter
64d1d82e03ace010dcf03c509cf6b87e0da27ff4
/users/budde377/encasa/tapas-survey/corpus/git/phpbb
e132e9ba76e63f71a6019b79c7a42417b3b633b4
/users/budde377/encasa/tapas-survey/corpus/git/symfony
e91058089c900071362270f0756b166bf4028edc
/users/budde377/encasa/tapas-survey/corpus/git/magento2
d0131d777e173f60c2f6acd435173b5c7bb513c5
/users/budde377/encasa/tapas-survey/corpus/git/zf2
97f67e0d06f1693a0b6e9e62561ca482adc85f7c


SVN Stuff (Wordpress)

[budde377@casa01 trunk] (master) \$ svn info
Path: .
Working Copy Root Path: /encasa/budde377/tapas-survey/corpus/svn/trunk
URL: https://develop.svn.wordpress.org/trunk
Relative URL: ^/trunk
Repository Root: https://develop.svn.wordpress.org
Repository UUID: 602fd350-edb4-49c9-b593-d223f7449a82
Revision: 31753
Node Kind: directory
Schedule: normal
Last Changed Author: SergeyBiryukov
Last Changed Rev: 31753
Last Changed Date: 2015-03-12 15:56:34 +0100 (Thu, 12 Mar 2015)



\end{comment}