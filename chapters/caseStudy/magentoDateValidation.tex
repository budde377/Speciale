\newpage
\section{Date validation}
Program \ref{lst:dateValidation} is from the e-commerce platform Magento 2, where the original functions were located in two different files. The \texttt{isValidDate} function takes three arguments expressing a date and returns a map being either empty or containing an error message at the \texttt{"invalidDate"} key.  The \texttt{isMinimumDay} similarly checks the validity of a date using the previous function. Here however, the output is a list which leads to a merge of a list and a map at line \ref{line:dateValidation-1}. 

\begin{program}
\begin{lstlisting}
<?php

function isValidDate($day, $month, $year) {
    $errors = [];
    if(!checkdate($month, $day, $year))
        $errors["invalidDate"] = "The given date is not valid";

    return $errors;
}

function isMinimumDay($day, $month, $year, $required, $minDay) {
    $errors = [];
    if($required && $day == 0 && $month == 0 && $year == 0)
        $errors[] = "The date is required";

    $result = isValidDate($day, $month, $year);
    $errors = array_merge($errors, $result); %* \label{line:dateValidation-1} *)

    if($minDay > $day)
        $errors[] = "The day should at least be " . $minDay;

    return $errors;
}
$valid = isMinimumDay(27, 5, 2015, true, 6);
$invalid = isMinimumDay(1, 3, 1991, true, 5);
\end{lstlisting}
\caption{Date validation example}
\label{lst:dateValidation}
\end{program}

The program uses two library functions; \texttt{checkdate}, modelled as a function returning boolean $\top$, and \texttt{array\_merge}, as described in section \ref{subsec:libraryFunctions}.




%The difference from the previous example is that the mixing occurs in a library function \texttt{array\_merge}. Each library function has transfer function which implements an abstract version of that particular function. The \texttt{array\_merge} abstract function also detects possible merges of different types of arrays and warns the user like shown in figure \ref{fig:dateScreenshot}.

Running the analysis on the program yields a single error, as illustrated in figure \ref{fig:dateScreenshot}. As opposed to the previous cases, this error reflects the main problem, namely the merge of a list and a map. It is not clear whether the issue is caused by a misunderstanding of the specification of the \texttt{isValidDate} function or is as designed.

\begin{figure}
\centering
\includegraphics[width=0.9\textwidth]{chapters/caseStudy/screens/date}
\caption{Running the analysis on date validation example}
\label{fig:dateScreenshot}
\end{figure}


Assuming the error occurs due to incorrect assumptions regarding the return value of \texttt{isValidDate}, the problem can be resolved by converting the map to a list. This is done in program \ref{lst:dateValidationFixed} using the \texttt{array\_values} library function. Running the analysis on this program raises no errors or warnings.


\begin{program}
\begin{lstlisting}
<?php

function isValidDate($day, $month, $year) {
    $errors = [];
    if(!checkdate($month, $day, $year))
        $errors["invalidDate"] = "The given date is not valid";

    return $errors;
}

function isMinimumDay($day, $month, $year, $required, $minDay) {
    $errors = [];
    if($required && $day == 0 && $month == 0 && $year == 0)
        $errors[] = "The date is required";

    $result = isValidDate($day, $month, $year);
    $errors = array_merge($errors, array_values($result));

    if($minDay > $day)
        $errors[] = "The day should at least be " . $minDay;

    return $errors;
}
$valid = isMinimumDay(27, 5, 2015, true, 6);
$invalid = isMinimumDay(1, 3, 1991, true, 5);
\end{lstlisting}
\caption{Date validation example}
\label{lst:dateValidationFixed}
\end{program}