\section{Joining array values}
This case is inspired by \todo{what?} and shows how the analysis yields an error when using the array function \texttt{array\_pop} on a map type array. The program seen in listing \ref{lst:arrayPopExample} builds a map from a person to an animal while also adding in a separator. The separators are then used when the map values are turned into a string using the \texttt{implode} function. The map can afterwards be used to refer to which animal belong to which person. To remove the unwanted separator at the end of the map \texttt{array\_pop} is used to remove the last element of the array. The analysis then alerts the user to the fact that \texttt{array\_pop} is used with a map which is suspicious. The separator should instead be specified as the first argument for \texttt{implode} which avoids the use of \texttt{array\_pop} altogether.

\begin{program}
\begin{lstlisting}
$people = ["John", "Jane", "Alice", "Bob"];
$animals = ["Dog", "Cat", "Bird", "Fish"];
$animalMap = [];

for($i = 0; $i < count($people); $i++) {
    $animalMap[$people[$i]] = $animals[$i];
    $animalMap[$i] = ", ";
}

array_pop($animalMap);
$animalString = implode("", $animalMap);
echo "The people have the following animals: " . $animalString;
\end{lstlisting}
\caption{Joining array values to a string}
\label{lst:arrayPopExample}
\end{program}

\begin{figure}
\centering
\includegraphics[width=0.9\textwidth]{chapters/caseStudy/newScreens/joinArray}
\caption{Using the pop function on a map array}
\label{fig:arrayPopScreenshot}
\end{figure}
\todo{Add sub-captions}

