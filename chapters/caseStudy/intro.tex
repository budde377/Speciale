In this chapter, the analysis from chapter \ref{ch:analysis} is evaluated on six small programs. The first program, evaluated in \ref{sec:monthnamenumber}, is a constructed example illustrating how the analysis handles the two array types introduced. The remaining five programs, evaluated in section \ref{sec:arraypivot} to \ref{sec:joiningarrayvalues}, have been found by manually inspecting the results from the dynamic analysis in chapter \ref{ch:study}, looking for uncategorizable arrays and misuse of list-operations, e.g.\ \texttt{array\_pop} on a map. 


When found, the programs were minimized (i.e.\ unnecessary logic was removed), rewritten to P0, and necessary library functions implemented. Running the analysis was performed with a context bound by length two ($k=2$). This bound only imposes a practical restriction on the recursive directory content program (section \ref{sec:directoryContent}), since the maximum possible depth of function calls for all other examples is two. 

Execution statistics (such as running time and memory consumption) have been omitted, since an implementation of a fast and resource efficient analysis is not a goal of this thesis. Instead, this chapter will investigate whether the analysis provides useful error-messages, which are of the right kind and if the problem can be resolved without raising new errors.



