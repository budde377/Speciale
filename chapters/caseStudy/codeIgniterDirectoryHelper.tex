\section{Directory content}
\label{sec:directoryContent}
Found in the CodeIgniter framework, program \ref{lst:directoryMap} demonstrates mixing the map and list type arrays. This program uses four library functions for traversing the file-system; \begin{enumerate*}[label=\itshape\roman*\upshape)]
\item \texttt{opendir} which creates a file-pointer for a given directory,
\item \texttt{readdir} which reads the name of the next file/directory in the given directory using the file-pointer,
\item \texttt{is\_dir} which as the name suggests checks if a path is a directory, and
\item \texttt{closedir} which \emph{closes} the file-pointer.
\end{enumerate*} Since P0 does not support file pointers, these are modelled as numbers in the analysis. 


\begin{program}
\begin{lstlisting}
<?php

function directory_map($source_dir, $directory_depth)
{
    if ($fp = opendir($source_dir))
    {
        $filedata = [];
        $new_depth	= $directory_depth - 1;
        while (FALSE !== ($file = readdir($fp)))
        {
            if (($directory_depth < 1 || $new_depth > 0) && is_dir($source_dir.$file))
            {
                $filedata[$file] = directory_map($source_dir.$file, $new_depth); %* \label{line:directoryMap-1} *)
            }
            else
            {
                $filedata[] = $file; %* \label{line:directoryMap-2} *)
            }
        }
        closedir($fp);
        return $filedata;
    }

    return FALSE;
}

$result = directory_map("testDir", 2);
\end{lstlisting}
\caption{Directory content example}
\label{lst:directoryMap}
\end{program}


The \texttt{directory\_map} function takes a directory name and a depth as input and returns the structure of the given directory recursively bound by the provided length. The resulting structure is a mix of a list of file names and a map from sub directory name to the structure of that sub directory. Mixing an array-list and an array-map, as done in program \ref{lst:directoryMap}, provides no obvious way of using the output of the function. Furthermore the program seems to assume that directories cannot have integers as names. If a directory has integer  $i$ as a name and more than $i$ files have been seen, then adding the new folder will override a file already added to the list. For example let a directory contain a file named \texttt{file1} and an empty sub directory named \texttt{0}. If \texttt{readdir} first returns the name of the file and then the folder, the result will be \texttt{[[]]} (a list containing an empty list). If the order is reversed, the result will be \texttt{["file1", []]}. This subtle assumption (or bug) is a result of \texttt{readdir} not specifying an order for the read operation, thus the user of the function cannot rely on a specific ordering of the files and directories being read.

This is a good example of how arrays in PHP serve as the go-to data-structure even when other alternatives might be better suited to handle the problem. In this case, representing the files and directories as objects would solve the problem.


The analysis raises three warnings on two code locations (see figure \ref{fig:directoryScreenshot}). Just as in the previous case, these warnings seem to have been raised as side-effects of the real problem, which is caused by the \texttt{\$filedata} array being represented as a map at line \ref{line:directoryMap-1}, a list at line \ref{line:directoryMap-2}, and joined at the end of the \texttt{if}-statement. This join yields the $\top$ array, which is the source of the last warning at both code-locations: \textit{"Target of write might be non-array"}, since it causes all heap locations to be updated when writing to \texttt{\$filedata}. Yielding a warning when a join results in a $\top$ array might be a better indicator of the \textit{real} problem.

\begin{figure}
\centering
\begin{subfigure}{\textwidth}
\centering
\includegraphics[width=0.9\textwidth]{chapters/caseStudy/screens/dir1}
\subcaption{Writing with string on list and type mismatch}
\label{fig:directoryScreenshot1}

\end{subfigure}
\begin{subfigure}{\textwidth}
\centering
\includegraphics[width=0.9\textwidth]{chapters/caseStudy/screens/dir2}
\subcaption{Appending a string and type mismatch}
\label{fig:directoryScreenshot2}

\end{subfigure}
\caption{Running analysis on \texttt{directory\_map}}
\label{fig:directoryScreenshot}
\end{figure}

As mentioned above, the files and folders could be represented as objects in order to solve the problem. However, since P0 does not support objects, these can be \textit{emulated} with arrays. In program \ref{lst:directoryMapFixed} the result is a map from file-names to maps representing files and folders. The file- and folder-array has key \texttt{"type"} pointing to \texttt{"file"} and \texttt{"dir"} respectively. If the \texttt{\$directory\_depth} is not exceeded, the folder-array also has key \texttt{"content"} pointing to a list representing the content of the folder, or \texttt{FALSE} if the folder could not be opened.  This solution will successfully represent a given folder independently of the order in which the folder content is read in by \texttt{readdir}.


\begin{program}
\begin{lstlisting}
<?php

function directory_map($source_dir, $directory_depth)
{
    if ($fp = opendir($source_dir))
    {
        $filedata = [];
        $new_depth	= $directory_depth - 1;
        while (FALSE !== ($file = readdir($fp)))
        {
            if (is_dir($source_dir.$file))
            {
                $dir = ['type'=>'dir'];
                if($directory_depth < 1 || $new_depth > 0){
                    $dir['content'] = directory_map($source_dir.$file, $new_depth);
                }
                $filedata[$file] = $dir;
            }
            else
            {
                $filedata[$file] = ['type'=>'file'];
            }
        }
        closedir($fp);
        return $filedata;
    }

    return FALSE;
}

$result = directory_map("testDir", 2);
\end{lstlisting}
\caption{Fixed directory content example}
\label{lst:directoryMapFixed}
\end{program}
